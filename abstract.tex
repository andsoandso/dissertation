\begin{abstract}

The neural mechanisms of reinforcement learning are becoming increasingly clear following years of exciting and intense inquiry.  However, due to their reliance on primary and secondary reward concepts, reinforcement learning theories can't account for two related facts.  One, rewarding effects are observed in the absence of primary and secondary reinforcers (e.g., novelty, information and fictive outcomes).  Two, value can be transferred by inference; no pairing is needed (e.g., stimulus generalization, optimistic firing).  These atypical, or ``cognitive rewards'', have received little direct investigation; this thesis examines then a proposed mechanism that could underlie both these facts -- by treating and modeling rewards as a kind of category, reward knowledge can be constructed and transferred (by similarity-based inference) to new situations.  Using behavioral, fMRI, and computational data, this proposal was tested.  Participants completed a stimulus-response task where classical rewards (e.g.,``Correct!'' or ``Win \$1.'') were replaced with pre-trained perceptual categories, one reward category for gains and one for losses.  The reward category for each trial was a unique, never before or again experienced examplar, distinguishing this task from higher conditioning experiments. In total, the behavioral and neural data strongly suggest that cognitive rewards are in fact categories, categories which do substantively impact fMRI-based reinforcement learning signals in the brains of the human participants.  It is then further argued that as category representations are a complete mechanistic explanation for the well established generalization of (classical) secondary reinforcers, rewards are categories -- which represents a substantial change in how rewards are conceived, and modeled: the primary, to secondary, to higher-order conditioning paradigm is incomplete, perhaps even incorrect.

\end{abstract}