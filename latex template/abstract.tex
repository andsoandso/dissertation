\begin{abstract}  % Required, Abstract, maximum of 350 words for Ph.D., 150 for MS/MA

% PUT YOUR TEXT BELOW HERE

It is widely accepted that increasing the number of aerosols available to act as cloud condensation nuclei (CCN) will have significant effects on cloud properties, both microphysical and dynamical.  This work focuses on the impacts of aerosols on deep convective clouds (DCCs), which experience more complicated responses than warm clouds due to their strong dynamical forcing and the presence of ice processes.  Several previous studies have seen that DCCs may be invigorated by increasing aerosols, though this is not the case in all scenarios.  The precipitation response to increased aerosol concentrations is also mixed.  Often precipitation is thought to decrease due to a less efficient warm rain process in polluted clouds, yet convective invigoration would lead to an overall increase in surface precipitation.  In this work, modeling and observations are both used in order to enhance our understanding regarding the effects of aerosols on DCCs.  Specifically, the area investigated is the tropical East Atlantic, where dust from the coast of Africa frequently is available to interact with convective storms over the ocean.

The first study investigates the effects of aerosols on tropical DCCs through the use of numerical modeling.  A series of large-scale, two-dimensional cloud-resolving model simulations was completed, differing only in the concentration of aerosols available to act as CCN.  Polluted simulations contained more deep convective clouds, wider storms, higher cloud tops and more convective precipitation across the entire domain.  Differences in the warm cloud microphysical processes were largely consistent with aerosol indirect theory, and the average precipitation produced in each DCC column decreased with increasing aerosol concentration.  A detailed microphysical budget analysis showed that the reduction in collision and coalescence largely dominated the trend in surface precipitation; however the production of rain through the melting of ice, though it also decreased, became more important as the aerosol concentration increased.  The DCCs in polluted simulations contained more frequent, stronger updrafts and downdrafts, but the average updraft speed decreased with increasing aerosols in DCCs above 6 km.  An examination of the buoyancy term of the vertical velocity equation demonstrates that the drag associated with condensate loading is an important factor in determining the average updraft strength.  The largest contributions to latent heating in DCCs were cloud nucleation and vapor deposition onto water and ice, but changes in latent heating were, on average, an order of magnitude smaller than those in the condensate loading term.  It is suggested that the average updraft is largely influenced by condensate loading in the more extensive stratiform regions of the polluted storms, while invigoration in the convective core leads to stronger updrafts and higher cloud tops.   

The goal of the second study was to examine observational data for evidence that would support the findings of the modeling work.  In order to do this, four years of CloudSat data were analyzed over a region of the East Atlantic, chosen for the similarity (in meteorology and the presence of aerosols) to the modeling study.  The satellite data were combined with information about aerosols taken from the output of a global transport model, and only those profiles fitting the definition of deep convective clouds were analyzed.  Overall, the cloud center of gravity, cloud top, rain top, and ice water path were all found to increase with increased aerosol loading.  These findings are in agreement with what was found in the modeling work, and are suggestive of convective invigoration with increased aerosols.  In order to separate environmental effects from that due to aerosols, the data were sorted by environmental convective available potential energy (CAPE) and lower tropospheric static stability (LTSS).  The aerosol effects were found to be largely independent of the environment.  A simple statistical test suggests that the difference between the cleanest and most polluted clouds sampled are significant, lending credence to the hypothesis of convective invigoration.  This is the first time evidence of deep convective invigoration has been demonstrated within a large region and over a long time period, and it is quite promising that there are many similarities between the modeling and observational results.


% DO NOT REMOVE
\end{abstract}
