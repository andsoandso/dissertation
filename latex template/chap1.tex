\section{General Introduction}

It has been known for many years that aerosols can impact important microphysical and dynamical properties of clouds  \citep{Twomey:1977p42,Albrecht:1989p347}; however, these effects are complicated and may depend on cloud type \citep{Seifert:2006p86,VanDenHeever:2011p7996}  and environment \citep{Khain:2008p35,Lebsock:2008p45,Fan:2009p7470,Storer:2010p8001}.  It has proven particularly difficult to understand the changes that occur in deep convective clouds (DCCs) for two main reasons: (1) the presence of ice in these clouds adds additional uncertainty to processes such as precipitation formation, and (2) DCCs can form in many types of environments in association with differing types of forcing.  Much of the research performed that has examined aerosol indirect effects on DCCs, as summarized by \citet{Khain:2009p18} and \citet{taoreview}, has had mixed results both in terms of the effect of aerosols on the precipitation produced by these clouds, as well as whether convective invigoration occurs in polluted storms.  The purpose of the work summarized here is therefore to add to the knowledge of how the presence of increased aerosols can impact DCCs, through a combination of modeling and observational analysis.

This dissertation is split into two main components.  Chapter 2 describes a modeling study undertaken in order to learn the important processes occurring in DCCs that are affected by increasing aerosol concentrations.  A series of large scale, two-dimensional model runs were completed, with only the aerosol concentration differing, and a large sample of DCCs was analyzed for aerosol effects.  The microphysical budget was examined in detail in order to distinguish which processes were important for precipitation formation and latent heating, and to examine how these processes were affected by the presence of increased numbers of aerosols.  This chapter has been submitted, in this form, as a manuscript to the Journal of Atmospheric Science, and has been accepted pending revisions.

\newpage\noindent
Up until now, no observational studies have been published in which aerosol impacts on deep convection are examined over a large spatial and temporal scale.  Other observational studies on aerosol indirect effects have focused on a very limited spatial domain, or did not isolate the impacts specifically on deep convection.  In order to evaluate whether the aerosol indirect effects described in the modeling study can be observed, a study was performed utilizing CloudSat data; this research is summarized in Chapter 3.  Four years of satellite data were analyzed in a region of the East Atlantic in keeping with the conditions simulated in Chapter 2.  A large sample of DCCs was acquired over a four year period and matched with aerosol optical depth (AOD) information obtained from a global transport model.  Four parameters (center of gravity, cloud top, rain top, and ice water path) were examined for differences associated with changes in AOD.  An attempt was made to separate out environmental effects from the effects of aerosols by splitting the data up by convective available potential energy (CAPE) and lower tropospheric static stability (LTSS), and the results were tested for significance.  This chapter is a manuscript in preparation to be submitted to a peer-reviewed journal.