\section{Observations of aerosol induced convective invigoration in the tropical East Atlantic}


\subsection{Introduction}
While increasing work has been done on the question of aerosol impacts on deep convection, many questions still remain unanswered.  Several studies \citep{Andreae:2004p32,Khain:2005p71,vandenHeever:2006p1147,vandenHeever:2007p53,Lee:2008p3014,Rosenfeld:2008p24,Lebo:2011p8933, rcepaper} suggest that increased aerosols will lead to the invigoration of deep convective storms, however some studies have had mixed results \citep[as summarized in][]{Khain:2009p18,taoreview}.  The theory for convective invigoration is as follows.  In an environment which contains more aerosols that can act as cloud condensation nuclei (CCN), clouds will produce less warm rain; this is due to the inefficiency of collision and coalescence when a cloud contains a large number of small drops.  In deep convective clouds, the reduced warm rain production leads to an increase in the amount of condensate in higher levels of the storms.  It is thought that increased freezing and vapor deposition then will provide enough excess latent heating to significantly increase the buoyancy of an updraft, thus leading to stronger storms with higher cloud tops, more ice, and heavier surface precipitation.

Observations of aerosol impacts on deep convection are hard to come by, but a few studies do exist to support the theory of convective invigoration suggested by modeling efforts.  The first observational evidence of convective invigoration was seen by \citet{Andreae:2004p32} in their study of convection over the Amazon during the biomass burning season.  Authors discovered strong thunderstorms with enhanced ice processes and heavy rain showers during smoke events, more so than when the environment was less polluted by smoke aerosols.  Others since \citep[e.g.][]{Lin:2006p54,TenHoeve:2012p8924} have also found evidence of higher cloud tops, enhanced heavy precipitation, and increased ice amounts in the same region.  Similar evidence has been found for convective invigoration (e.g. increased cloud tops, more frequent lightning, and heavier precipitation events) in polluted environments in the southeast United States \citep{Bell:2008p8489} and in China \citep{Wang:2011p8523}.  \citet{Wang:2009p8935} analyzed both observational and modeling data and found evidence that smoke from biomass burning in Central America may lead to an enhancement of severe weather in the south central United States.  \citet{Heiblum:2012p8926} recently performed an observational study examining satellite data over several regions across the globe.  They found evidence of convective invigoration in many of the regions studied, in the form of increased rain center of gravity.

In a recent study, \citet{rcepaper} attempted to look at the questions surrounding aerosol impacts on deep convection by utilizing a series of large-domain cloud resolving model simulations over the tropics.  They looked specifically at deep convective clouds and examined a detailed microphysical budget in order to help isolate important processes that were affected by increased aerosols.  The response of updraft speed to increased aerosols was not clear, due to balances between increased buoyancy from latent heat release and the drag from the increased condensate.  However, the authors found that storms formed in polluted simulations were more numerous, larger in horizontal extent, had higher cloud tops, and produced more total precipitation; that is, they found significant evidence for the theory of convective invigoration.  The authors concluded that more observations are necessary to understand the full story of how aerosols impact deep convective storms.  

The goal of this study is to examine a large sample of tropical deep convective clouds in order to assess the likelihood of the convective invigoration mechanisms proposed in \citet{rcepaper}.  We utilize satellite observations of deep convective clouds from the CloudSat Cloud Profiling Radar \citep[CPR,][]{Stephens:2002p1094}, in combination with global model output data used as a proxy for aerosol loading, in order to look for evidence of convective invigoration.  The use of CloudSat data provides a unique opportunity to examine aerosol impacts on deep convection because of the global coverage of the satellite and the ability of the radar to penetrate clouds, offering information about vertical structure.  In keeping with previous works, we will show that occurrences of deep convection in more polluted environments will have greater reflectivity throughout the column due to larger amounts of (ice and liquid) condensate throughout the cloud, and the clouds will be more vertically developed, due to convective invigoration.


\subsection{Data and Methods}

Observations of deep convection were obtained from the CloudSat Cloud Profiling Radar \citep[CPR,][]{Stephens:2002p1094}.  The CPR is a 94 GHz nadir-looking radar with a vertical gate spacing of 240 m. The horizontal resolution is 1.4 km (cross track) x 2.5 km (along track). The high frequency of this radar allows for high sensitivity to cloud and ice particles (the minimum detectable signal is about -30 dbZ). Reflectivity data used in this study is obtained from the level-2 product 2B-GEOPROF, which provides vertical profiles of reflectivity corrected for gaseous attenuation.

For the purposes of this study, only profiles selected as ``Deep Convective Clouds'' (DCCs) were analyzed.  A DCC was defined where the depth of continuous cloud was at least 8km; it is a similar selection to that used in \citet{rcepaper}.  These do not represent separate clouds, but are individual profiles measured by the CPR, many of which may be present in a large deep convective storm.  The sample of DCCs may consist of both the convective core of storms as well as regions of more stratiform-like cloud.  

Four years (2006-2009) of CloudSat data were analyzed for the region shown in Figure \ref{fig1}.  This region of the East Atlantic was chosen due to the frequent dust events that occur off the west coast of Africa \citep{carlsonsal,Zipser:2009p7664}.  In addition, much of the aerosol present in this region is likely to be composed of dust and sulfate \citep{Zipser:2009p7664}, similar to that modeled in \cite{rcepaper}.  Also shown in Figure \ref{fig1} is the frequency of occurrence of DCCs in the region.  Contoured is the percent of the total sample located in a 1$^{\circ}$ by 1$^{\circ}$ box.  A majority of the DCCs analyzed are located in the lower latitudes, likely related to the Intertropical Convergence Zone (ITCZ).  Note that this is not a cloud fraction; however the locations of more frequent DCCs align well with what is shown in \citet{Liu:2007p74}.

In order to examine aerosol indirect effects, a large, consistent record of aerosol measurements was required.  Satellite aerosol optical depth measurements collocated with deep convection are difficult to obtain, as most aerosol algorithms require an absence of cloud, and few in situ data exist over the ocean.  For these reasons, output from a global model was used to determine if a DCC was formed in a polluted or clean environment.  The Global and Regional Earth-System Monitoring Using Satellite and In situ Data (GEMS) project \citep{Hollingsworth:2008p8503} utilizes data assimilation of emissions inventories and satellite data, in combination with the ECMWF model, in order to provide detailed global coverage of chemical and aerosol species.  The project provides model simulated 550 nm aerosol optical depth (AOD), representing what would be observed by a sensor such as MODIS \citep[Moderate Resolution Imaging Spectroradiometer,][]{Remer:2005p3445}. GEMS model output was matched to CloudSat profiles, and the DCCs were divided into 10 groups of equal size based on the AOD.  Various DCC properties, as described below, were calculated for each profile and then averaged within each aerosol bin.  

A center of gravity (COG) was calculated for each profile utilizing a technique similar to that described by \citet{Koren:2009p8934}, and utilized by \citet{Heiblum:2012p8926}; however, instead of rain rate, the COG was calculated using values of reflectivity, as shown below.

\begin{equation}
COG=\frac {\sum _{i}R_{i}H_{i}} {\sum _{i}R_{i}}
\end{equation}

In this equation, R is the measured reflectivity and H is the height in meters of each level, i.  The sums performed in the calculation of COG began at the level of maximum reflectivity, rather than the surface, in order to lessen the possibility that attenuated profiles were affecting the results.  Generally a higher COG is present where values of reflectivity are greater, or more mass is present at higher levels of the cloud.
\newpage
Cloud top was defined as the highest level containing cloud, and rain top is the highest level where the reflectivity has a value of at least 0 dBZ.  To learn something of the microphysical characteristics of these clouds, ice water content was obtained from the CloudSat Level 2 data product 2B-CWC-RO \citep{Austin:2009p4345}.  The ice water content was vertically integrated to create ice water path.  

As previous work has found that aerosol indirect effects can differ depending on such environmental parameters as convective available potential energy \citep[CAPE, e.g.][]{Storer:2010p8001} and lower tropospheric static stability \citep[LTSS, e.g.][]{Matsui:2004p761}, it was useful to separate aerosol results by environmental regimes.  Environmental variables were calculated using information from the ECMWF-AUX product, and CAPE and LTSS (simply the lapse rate between the surface and the 700 mb level) were used to separate environmental impacts from those of aerosols on convective storms.

A simple monte carlo method was employed to test the significance of differences between the ``clean'' and ``polluted'' profiles.  For this purpose, ``clean'' and ``polluted'' profiles were considered those in the lowest and highest AOD bins respectively.  In each significance test, only the relevant  ``population'' was considered.  For example, a test was done for each population, where the population was all DCCs, only those with high CAPE, only those with medium LTSS, etc.
To perform the test, two random samples were pulled from the population, the same size as the clean and polluted samples within that population, and the difference in the means was recorded.  After 10,000 iterations, a difference between the clean and polluted samples is considered significant if less than 5\% of the tests produced a difference with a magnitude larger than the difference between clean and polluted.


\subsection{Results}
Nearly half a million DCCs were analyzed over a four year period.  As explained in Section 3.2, DCCs are those profiles with cloud present through a depth of at least 8 km.  For a sense of what these DCC reflectivity profiles look like, Figure \ref{fig2} shows a contoured frequency by altitude diagram \citep[CFAD, ][]{Yuter:1995p8937} of the entire sample of DCCs analyzed in this study.  The average cloud top of the DCCs analyzed is just over 12.1 km; above this, the majority of the DCCs have reflectivity values near or below the minimum detectable signal.  Moving downward, the reflectivity quickly increases as the CPR detects ice particles in the upper levels of the clouds.  Nearly all of the profiles sampled have a maximum reflectivity greater than 0 dBZ, which is often considered the threshold for precipitation sized particles, and a large number reach 15 dBZ, signifying strong vertical motion and large amounts of condensate.  An increase in reflectivity associated with the melting level is clearly visible near the 4 km level, and below that the reflectivity decreases rapidly towards the surface as attenuation of the radar signal due to precipitation becomes a factor.

The AOD estimated in the geographic region studied here ranged between 0.03 and 1.64 during the time period studied.  A histogram of AOD during the analysis period is shown in Figure \ref{fig3}.  As described above, in order to assess the impact of aerosols on the properties of DCCs, the entire population was divided into 10 bins of equal sample size based on the AOD.  Also indicated on Figure \ref{fig3} are the divisions between the 10 aerosols bins.  The average AOD in each bin was chosen as a representative value for the purpose of plotting trends in DCC characteristics.  The AOD in this region is widely variable.  Aerosol climatologies such as \citet{Remer:2008p8938} have shown that remote ocean regions typically have average AOD on the order of 0.1, while strongly polluted regions such as the Amazon during the biomass burning season have average values of AOD approaching 1.  However, the majority of DCCs sampled have AOD values that fall within a more ``moderate'' range of 0.2-0.4.

The first DCC property analyzed here is the center of gravity, as defined in Section 3.2.  The COG represents both the magnitude of reflectivity values and how high they reach in the troposphere.  Figure \ref{fig4}a displays the average value of COG for each aerosol bin; there is a clear upward trend in COG with increasing AOD.  Higher values of COG can result from clouds with larger vertical extent and/or clouds containing more hydrometeors, both of which have been seen in polluted storms \citep[as summarized in][]{taoreview}.  Larger vertical extent would suggest possible convective invigoration, as seen previously \citep{Andreae:2004p32,Khain:2005p71,vandenHeever:2006p1147,vandenHeever:2007p53,Lee:2008p3014,Rosenfeld:2008p24,Lebo:2011p8933, rcepaper}, while many studies \citep{Lynn:2005p29, Khain:2005p71,vandenHeever:2006p1147, VanDenHeever:2011p7996, Storer:2010p8001,rcepaper} have seen that increased aerosol loading leads to enhanced ice and liquid amounts in deep convective clouds.  \citet{Heiblum:2012p8926} also saw increased values of COG with increased aerosols in their study utilizing TRMM data.

In order to assess possible reasons for the increase in COG with increased aerosol loading, Figure \ref{fig4} also demonstrates the trends in cloud top, rain top (0 dbz echo height), and ice water path.  Cloud top and rain top show clear increases with increased aerosols, which suggests that convective invigoration may be occurring with higher values of AOD.  This signal of increased cloud and rain top would appear to be in keeping with results from previous studies (as discussed above) that found deeper and larger storms in more polluted simulations.

The ice water path decreases for moderate values of AOD and then increases substantially as aerosol loading is increased further.  An increase in cloud ice in polluted storms has been seen in several modeling studies  \citep{Lynn:2005p29, Khain:2005p71,vandenHeever:2006p1147, VanDenHeever:2011p7996,Rosenfeld:2008p24,Storer:2010p8001,rcepaper}.  It has been suggested in these studies that a reduction in precipitation efficiency in polluted clouds leads to higher amounts of cloud water that can then be lofted into the mixed phase region and result in increased cloud ice.   We hypothesize here that a similar feedback may be happening for the cleanest environments; that is, environments that are too pristine may have less efficient precipitation production if there is a lack of dust or other cloud-nucleating particles that would aid in promoting cloud drop growth.  Thus, environments with moderate values of AOD may have the most efficient rainout and therefore less cloud water available for lofting, resulting in the lowest ice amounts.

As mentioned in Section 3.2, the net effect of aerosols on convection can depend strongly on the environment, and thus, the DCCs analyzed were separated by environmental parameters in an attempt to isolate the signal attributable to aerosol loading.  Histograms of CAPE and LTSS for the region and time period analyzed here are shown in Figure \ref{fig5}.  The total sample was split into three equal sized bins by each of the environmental characteristics (high, medium, and low) and the same divisions were used within each aerosol bin.  The same four DCC characteristics discussed previously are plotted in Figure \ref{fig6}, split into high, medium, and low CAPE.  It is clear that all four properties increase with increased CAPE.  This is not surprising, as higher values of CAPE are typically associated with stronger, more vertically developed storms which would also contain more ice.  Trends in COG are less clear when split up by environment; the presence of some amount of CAPE seems to be the driving factor for moving large amounts of mass to high levels of the storms.  However, aerosol-induced trends in cloud top, rain top, and ice water path clearly hold for all types of environment.  Similar to the results seen in \citet{Storer:2010p8001}, CAPE is the dominating factor when determining these properties, yet the aerosol loading has a notable effect.  The trends are generally clearer for the low and medium CAPE bins, which also follows with \citet{Storer:2010p8001} who saw decreased impacts of aerosols for storms in the most unstable environments.

Figure \ref{fig7} demonstrates that results are similar when DCCs are separated by LTSS, rather than CAPE.  Though not as clear as with CAPE, the four properties examined here (COG, cloud top, rain top, and ice water path) all generally increase with increasing LTSS.  It may seem counterintuitive that such convective characteristics would increase with increased lower level stability.  However, it can be explained in the following way: high values of LTSS can act as a cap to suppress convection, but the presence of some stability in the lower levels can help to prevent widespread convective initiation such that convection that does form is strongly forced.  Regardless of the value of LTSS, the DCC characteristics demonstrate the same trends of increasing with increased AOD.

While the trends in the parameters analyzed are subjectively clear, there is is quite a bit of noise in a few of the plots, particularly in the moderate AOD range of 0.2-0.4.  \citet{rcepaper} similarly found strong relationships when considering very clean and very polluted conditions, but saw increased variability in the aerosol response for moderately polluted conditions.  In order to objectively determine whether the aerosol effect is indeed significant, a test was performed on each sample of DCCs as described in Section 3.2.  For the purposes of significance testing, ``clean'' is considered to be the lowest aerosol bin, and ``polluted'' the highest.  Table \ref{sigtable} shows the difference between the polluted and clean DCCs for all of the characteristics analyzed.  In bold are the values which are significant at the 95\% level.  This demonstrates that most of the trends discussed are significant and cannot be attributed to noise in the sample.  Cloud top is the clearest trend, with all differences significant at the 95\% level, and rain top is nearly as consistent, as those values which are not significant at the 95\% level in the rain top column would be considered significant if the threshold were lowered to 90\%.  All but one of the trends in center of gravity are significant at the 95\% level.  Ice water path is less consistent than the other parameters; however, given the shape of the trend discussed above, this may not be surprising.  That is, there is likely a physical reason for the cleanest sample to have similar ice amounts to the most polluted.   

To visualize the differences due to aerosol effects, a CFAD was calculated for the difference between the polluted and clean DCCs (Figure \ref{fig8}a).  To create this figure, a CFAD was created for each sample and normalized before the difference was taken.  A distinct difference between the clean and polluted samples can be noted here, with a shift toward larger values of reflectivity higher in the atmosphere in the polluted DCCs.  In addition, a histogram of cloud top and rain top was contoured in Figure \ref{fig8}b.  This difference plot was calculated in a similar manner as the difference CFAD and also demonstrates a clear shift toward higher cloud tops and rain tops for those DCCs in the most polluted environments. 


\subsection{Discussion}
The evidence presented from 4 years of satellite data indicates that increased aerosol loading has significant impacts on the structure of deep convective clouds.  The results support previous observational and modeling evidence \citep{Andreae:2004p32,Khain:2005p71,vandenHeever:2006p1147,vandenHeever:2007p53,Lee:2008p3014,Rosenfeld:2008p24,Lebo:2011p8933, rcepaper} that increased aerosols can lead to convective invigoration.  As discussed in Section 3.1, \citet{rcepaper} provide a detailed explanation for how convective invigoration can occur in deep convective clouds.  Polluted clouds produce much less rain through the processes of collision and coalescence, as the large numbers of small droplets make the warm rain process inefficient.  There is then much more liquid water remaining in the clouds, which can be lofted to form ice.  This mechanism can explain the increased ice water path that was seen in the observations summarized here.  Since there is more condensate in the polluted clouds, more latent heat is released in the freezing of liquid cloud drops into ice.  Additionally, as there exist more and smaller hydrometeors in the polluted clouds, there is highly increased surface area onto which vapor can deposit.  Hence, vapor deposition onto both vapor and ice increases substantially, adding also to increased latent heating throughout the cloud.  The increased buoyancy from latent heating competes with the increased weight of condensate loading the updraft, but \citet{rcepaper} found that despite this balance, DCCs in polluted simulations were deeper, wider in horizontal extent, and produced more total precipitation.  The fact that observational evidence shown here is in agreement with these findings suggests that these mechanisms described by \citet{rcepaper} (and also others, as summarized above and by \citet{taoreview}) may be an explanation for the signals of convective invigoration seen here.

This particular region of the globe was chosen for study because of the frequent dust storms that occur off the west coast of Africa, and also for the similarity in environment and aerosol type to that modeled by \citet{rcepaper}.  The aerosols in this region are typically dust and sulfate aerosols.  Other regions may be characterized by the presence of such aerosols as black carbon, which can absorb significant amounts of solar energy and change the possible dynamics in play.  Also, as convection is not identical across the world, meteorological impacts on convection may change how it can respond to the presence of increased aerosols.  For instance, several studies \citep[e.g.][]{Khain:2010p8508, Storer:2010p8001} showed that strongly forced midlatitude convection may not show a signal of convective invigoration.  It is thus important to note that this study may not be representative of other regions, or other aerosol types.  

Much work still needs to be done on the problem of convective invigoration, but this study shows some evidence that the theory of convective invigoration may be verifiable in at least one region.


%FIgure 1
\begin{figure}[t]
 \noindent\includegraphics[width=39pc]{map.pdf}
 \caption{The region analyzed in this study is shown here.  In the inset, the frequency of occurrence of deep convective clouds is contoured.}\label{fig1}
 \end{figure}
 
 %Figure 2
\begin{figure}[t]
 \noindent\includegraphics[width=39pc]{allcfad.pdf}
 \caption{A contoured frequency by altitude diagram (CFAD), showing the frequency of occurrence of values of reflectivity at different heights for the total sample of deep convective clouds sampled in this study.}\label{fig2}
 \end{figure} 
 
 %Figure 3
 \begin{figure}[t]
 \noindent\includegraphics[width=39pc]{aodhistogram.pdf}
 \caption{A histogram showing the frequency of occurrence of values of aerosol optical depth.  Vertical dashed lines denote the divisions between the ten aerosol bins used to split up the data.}\label{fig3}
 \end{figure}
 
 %Figure 4
 \begin{figure}[t]
 \noindent\includegraphics[width=39pc]{allfourpanel.pdf}
 \caption{Average values of a) center of gravity, b) cloud top, c) rain top, and d) ice water path for each aerosol bin.}\label{fig4}
 \end{figure}
 
 %Figure 5
 \begin{figure}[t]
 \noindent\includegraphics[width=39pc]{capeltsshistogram.pdf}
 \caption{Histograms showing the frequency of occurrence of a) convective available potential energy (CAPE) and b) lower tropospheric static stability.  Vertical dashed lines denote the divisions between high, medium, and low values for each environmental parameter.}\label{fig5}
 \end{figure}
 
 %Figure 6
 \begin{figure}[t]
 \noindent\includegraphics[width=39pc]{capefourpanel.pdf}
 \caption{Average values of a) center of gravity, b) cloud top, c) rain top, and d) ice water path for each aerosol bin.  Deep convective clouds are split by high, medium, and low CAPE.}\label{fig6}
 \end{figure}
 
 %Figure 7
 \begin{figure}[t]
 \noindent\includegraphics[width=39pc]{ltssfourpanel.pdf}
 \caption{As in Figure 6, but profiles are split up by LTSS.}\label{fig7}
 \end{figure} 
 
 %Figure 8
  \begin{figure}[t]
 \noindent\includegraphics[width=39pc]{ctrtcfad.pdf}
 \caption{a) A difference CFAD, calculated by subtracting normalized CFADs from the most polluted and cleanest aerosol bins.  b) A two-dimensional histogram showing the difference in the frequency of occurrence of values of cloud top and rain top between the polluted and clean samples.}\label{fig8}
 \end{figure}
 
% \end{document}
%
% \begin{table}
% \caption{}
% \end{table}
%
% ---------------
% TWO-COLUMN figure/table
%
% \begin{figure*}
% \noindent\includegraphics[width=39pc]{map.pdf}
% \caption{Caption text here}
% \end{figure*}
%
% \begin{table*}
% \caption{Caption text here}
% \end{table*}
%
% ---------------
% EXAMPLE TABLE
%
%\begin{table}
%\caption{Time of the Transition Between Phase 1 and Phase 2\tablenotemark{a}}
%\centering
%\begin{tabular}{l c}
%\hline
% Run  & Time (min)  \\
%\hline
%  $l1$  & 260   \\
%  $l2$  & 300   \\
%  $l3$  & 340   \\
%  $h1$  & 270   \\
%  $h2$  & 250   \\
%  $h3$  & 380   \\
%  $r1$  & 370   \\
%  $r2$  & 390   \\
%\hline
%\end{tabular}
%\tablenotetext{a}{Footnote text here.}
%\end{table}

% See below for how to make landscape/sideways figures or tables.

\clearpage

\begin{table}[t]
\caption{Difference between ``Polluted'' and ``Clean'' DCCs for each sample.  Values that are significant at the 95\% level are in bold}\label{sigtable}
\begin{center}
\begin{tabular}{l | cccc}
\hline
 & COG (m) & Ice Water Path (g/m$^2$) & Cloud Top (m) & Rain Top (m) \\
\hline
All DCCs & \textbf{437.8} & \textbf{202.0} & \textbf{766.0} & \textbf{829.0} \\
High CAPE &  \textbf{273.4} & -54.2 & \textbf{502.4} & \textbf{395.3}  \\
Medium CAPE  &  \textbf{418.5}  &  \textbf{304.1}  &  \textbf{691.4}  &  \textbf{854.3}  \\
Low CAPE  &  \textbf{323.0}  &  \textbf{166.6}  &  \textbf{572.2}  &  \textbf{723.8}    \\
High LTSS  &  \textbf{271.1}  &  \textbf{422.7}  &  \textbf{281.0}  &  \textbf{698.4}  \\
Medium LTSS  &  221.9  &  115.7  &  \textbf{419.6}  &  542.2  \\ 
Low LTSS  &  \textbf{601.0}  &  17.7  &  \textbf{1185.2}  &  871.6\\
\hline
\end{tabular}
\end{center}
\end{table}

