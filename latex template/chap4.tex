\section{General Conclusions}

\subsection{Summary}
This dissertation describes a combination of modeling and observational work completed with the goal of gaining new understanding of aerosol indirect effects on deep convection.  The modeling study examined in detail the microphysical processes in deep convective storms impacted by the presence of aerosols, and the observational study provided the opportunity to examine a large sample of deep convective clouds for evidence that the mechanisms described in the modeling study may be, in fact, occurring in the region analyzed.

\subsubsection{Model Simulations}
A series of large-scale simulations of the tropical atmosphere were completed using a cloud-resolving model run using the framework of radiative-convective equilibrium.  Only those model columns that fit the definition of a DCC were analyzed, for a total of ten days of simulation time.  The only difference between the simulations was the concentration of aerosols available to act as CCN.  With an increase in aerosol concentration, simulations contained more DCCs, which produced more total convective precipitation.  The average storm width was also greater for higher aerosol concentrations, and there were more high cloud tops.  These domain-wide statistics suggest convective invigoration with increased aerosol concentration.

The general microphysical characteristics of the DCCs varied with increasing numbers of aerosols as predicted by previous work on aerosol indirect effects.  Polluted DCCs had more cloud drops that had, on average, smaller diameters.  Theory predicts that collision and coalescence will be less efficient as cloud drop size decreases, leading to a reduction in the warm rain process.  As predicted, the DCCs had, on average, less rain, but increased cloud water and ice remained within the clouds.  
\newpage
As rain can be formed both through collision/coalescence and through the melting and shedding of hail and other ice hydrometeors, an analysis was performed using microphysical budgeting terms, in order to determine the primary effects of aerosols on surface precipitation.  There are three terms which lead to the production of rain in the model: cloud to rain (the warm rain process), ice to rain (the collection of cloud ice by rain), and the melting and shedding of hail.  The cloud to rain term decreases dramatically with increased aerosol concentration, as would be expected.  The collection of ice also decreases, as it is a function of how much rain exists to collect.  The melting of hail does not change significantly with increased aerosol concentrations.  As aerosol concentrations increase, the reduction in warm rain efficiency dominates the trend in surface precipitation.  Also the ice phase production of precipitation becomes more important in the more polluted simulations, because the warm rain process has diminished enough that melting of hail in addition to the collection of ice totals to more precipitation production than that from warm rain.  The changing importance of ice phase precipitation processes with increased aerosol concentration suggests that the decrease in average rain production seen here may not be consistent for midlatitude storms, as these storms generally have a more developed ice phase which may shift the trends in precipitation efficiency.  The precipitation response is complex, as it depends on a balance of collision/coalescence, melting, collection, and also the evaporation of rain.  

Since the domain-wide statistics suggest that convective invigoration occurs with increased aerosol concentration, an analysis of vertical velocity was completed to determine whether the results were in agreement with the domain-wide trends.  A histogram of vertical velocity showed that there were more higher values of w in polluted DCCs, yet significantly fewer moderate values.  A profile of average updraft speed revealed that the average updraft actually decreased with increasing aerosol concentrations.  In order to explain this decrease, the buoyancy term of the vertical velocity equation was calculated.  This term has two parts: the increase in buoyancy due to latent heat release, and the decrease in buoyancy resulting from the drag of condensate loading.  The trend in average buoyancy was dominated by the condensate loading term, suggesting that on average, the increased condensate loading in DCCs was greater than any increase in latent heating.  The latent heating term itself did not demonstrate clear trends with increased concentrations of aerosols.  This is due to the fact that the net latent heating is a balance between many processes; for instance, condensation and vapor deposition onto ice increase with increased aerosol concentrations, as does the evaporation of cloud water, which has the opposite sign.  The effect of aerosols on vertical velocity will depend on the balance of all of the processes discussed here.  It appears that the effects are mixed in these simulations, with some DCCs experiencing convective invigoration, but not all of them.  It may be that some storms, and not others, experience convective invigoration.  Alternatively, it may be that storms undergo invigoration early in their life cycle, but not later when the DCCs are more stratiform in nature.  While the exact effect of increased aerosol concentration on vertical velocity is complex, and more work must be done in the future in order to determine the exact processes at play, the model simulations do show strong evidence of convective invigoration for increased aerosol concentrations.  

\subsubsection{Satellite Observations}

A large-domain observational study was conducted in order to evaluate whether the mechanisms described in the modeling study can be observed in actual DCCs.  Four years of CloudSat data were combined with AOD information from a global transport model to investigate aerosol indirect effects on DCCs in the East Atlantic.  Four DCC parameters (center of gravity, cloud top, rain top, and ice water path) were calculated and binned by AOD in order to search for trends.  All four parameters were seen to increase with increasing aerosol concentration.  The increase in center of gravity, cloud top, and rain top all indicate convective invigoration, as they demonstrate that more mass has moved to higher levels of the atmosphere.  The increase in ice is consistent with theory and previous results, as increased condensate is typically seen due to the reduction in the warm rain process in polluted clouds.

As previous studies have found that aerosol indirect effects can depend on the environment, the DCCs were split up by CAPE and LTSS.  In this way, trends in cloud properties due to AOD could be isolated from those due to the environment.  All of the cloud parameters increased with increasing CAPE and LTSS; also, the trends in the cloud properties with AOD still held for most values of environmental parameters.  The trends in DCC parameters with AOD were more consistent for lower values of CAPE; this is in keeping with previous work that has suggested that aerosol indirect effects will be less noticeable for convection forming in more unstable environment, as the environmental forcing will be much stronger.  

The trends in DCC parameters with AOD exhibit noise within the range of AOD that could be considered moderate, that is 0.2-0.4.  Similarly, in the modeling study described in Chapter 2, trends were found to be clearest when comparing the cleanest and most polluted simulations, with some variability in moderate values of aerosol concentration likely due to the complex interaction of the various microphysical processes affected.  To determine whether the trends observed here are significant, a simple monte carlo test was completed.  This test showed that most of the differences between the cleanest and most polluted DCCs are significant at the 95\% level.  This is the first study that examines aerosol indirect effects on deep convection over a fairly large spatial and temporal domain, and that the results are in agreement with what has been seen in model simulations shows promise that the mechanisms described by the model can explain what is occurring.  


\subsection{Conclusions and Recommendations for Future Work}

Much more work needs to be done before we have fully solved this complex problem.  In terms of modeling work, the budgeting analysis utilized in Chapter 2 is a very useful tool.  It would be helpful to repeat such an analysis in further detail for more simulations, in order to test in more detail how other factors influence the complex balance of various microphysical processes.  One subject of interest would be how processes change in different environments: that is, midlatitude instead of tropical, different environmental profiles, or including other factors such as shear.  
\newpage
More information is also necessary about how aerosol effects vary in the more moderate range of aerosols.  As shown in Chapter 2, the net effect of aerosols on DCCs depends on the balance of various microphysical processes.  The modeling study presented in Chapter 2 used doubling of aerosol concentrations in order to investigate a wide range, however it will be necessary to examine smaller changes in aerosol concentration within a moderate range (e.g. 500 - 1000 cm$^{-3}$).  Within this range, it is likely that competing processes are adding variability to the net effect of aerosols, and more study is required to understand this variability.  Single cloud studies would also be useful, as it would allow for examining time series of microphysical budgeting terms, leading to a greater understanding of when and how convective invigoration occurs.  

More observational work needs to be done, especially now that we have such a dataset as CloudSat, which has good global coverage and a lot of information about deep convection.  Similar analysis to that presented here can be done in various regions around the world and compared to these results.  In terms of observational data which would help aerosol studies, it would be useful to have better information about aerosol composition, as, for instance, black carbon aerosols will likely behave significantly differently than sulfate.  Also, much better observations of cloud processes are needed in order to truly observe the details of aerosol indirect effects, and to improve modeling studies thereof.

While the problem is not solved, the work presented here shows promising new insight into the problem of how aerosols impact deep convective clouds.  Detailed microphysical budgets have provided information about the processes impacted by the presence of aerosols, and observations have demonstrated promising evidence that supports the conclusions of the modeling work.
