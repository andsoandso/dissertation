\documentclass[doc,12pt]{apa}        % use: 'man' for submission type; 'jou' for
                                % journal type, and 'doc' for typical latex
                                % but with figures inline with text
\usepackage{geometry} 
%\geometry{a4paper} 
\usepackage[parfill]{parskip}   % paragraphs delimited by an empty line

\usepackage{graphicx} 
\usepackage{amssymb}            % no idea what this does...
\usepackage{epstopdf}           % no idea what this does...
%\usepackage{gensymb}            % no idea what this does...

\usepackage{setspace}

\DeclareGraphicsRule{.tif}{png}{.png}{`convert #1 `dirname #1`/`basename #1 .tif`.png} \setcounter{secnumdepth}{0}  % no idea what this does...

\usepackage{apacite}
%%%%%%%%% END HEADER %%%%%%%%%

\title{Rewards are categories.} 
\author{Erik J. Peterson} \affiliation{Dept. of Psychology \\ Colorado State University \\ Fort Collins, CO} 

%%%%%%%%%%%%%%%%
\begin{document} 
%%%%%%%%%%%%%%%%
\maketitle
%\doublespacing

\section{A Task and Some Models} % (fold)
\label{sec:task_and_models}

\subsection{On task}
\label{to_task}
\subsubsection{What they did and when}
\label{what_when}
The task consisted of two parts.  Depicted in Fig~\ref{fig:task}. (top), the first is passive wherein participants learned the two reward categories by viewing randomly selected black and white sinusoidal gratings followed by ``Gain \$1'' or ``Lose \$1'' in, respectively, green or red letters.  The shape of the gratings for each category was derived from a information integration parameter distribution (Fig. \ref{fig:II}), borrowed from \cite{Spiering:2008p5008}).  Each grating was on-screen for 2 seconds, followed by 1 seconds gap and an empty grey screen, with the outcome displayed for 1 seconds with a 0.5 fixation cross between each trials. Each trial lasted a total of 4.5 seconds. Part 1 was spread over an initial training period lasting 126 trials, and an in-scanner refresher lasting 45 trials.  Prior to beginning training participants were instructed to ``Attend to the screen in order to learn which types of gratings indicate wins and which types indicate losses''.  The category parameter distribution (Figure. \ref{fig:II}) to reward (i.e. gain or loss) mapping was randomized for each participant.

Part 2 is a deterministic unstructured stimulus-response task that replaces direct verbal feedback or reward with an appropriate grating from task 1  (Figure. \ref{fig:task}, \emph{bottom}).  Gratings matching monetary the Gain category were used for positive reinforcement, while and gratings indicative of losses were used as negative reinforcers.   Each trial began with an abstract black and white ``tree'' stimuli (left most image in Figure. \ref{fig:task}), which belonged to one of two response categories (``q'' or ``w'').  Subjects indicated their response by button press using either the right index (``q'') or left(``w'') on a magnet compatible response box.  The response window lasted up to 2.5 seconds, but was terminated as soon as a response was made.  Immediately following response the ``tree'' was replaced with a blank grey screen, which was on-screen for half a second and was replaced with a feedback screen.  If the response was correct a new, that is never before experienced, exemplar grating from the Gain distribution is used; if the participant was incorrect, a new Loss grating appeared instead.  If no response was made or the wrong button was pressed the subject instead saw, ``No response detected.'', printed on-screen.  Feedback stayed on-screen for 1 second and was then by a fixation cross for 0.5 seconds.  For this part participants were instructed to, ``Use what you learned about the rewarding properties of the gratings to try and earn as much money as possible in this portion of the experiment.''.  Instruction for both parts were given orally by the experimenter using a script, using Figure. \ref{fig:task} as a visual aid.

Over the course of part 2, participants learned to classify 6 ``trees'', randomly selected at the start of the experiment out of a pool of 22 possible.  Each of the 6 were experienced a total of 28-32 times for a total of 199 trials. The order of the trials was determined using a genetic algorithm designed to optimize fMRI signal detection, among other considerations.  Most relevant to behavioral analysis, trials were in pseudo-random order with second order counterbalancing.  For complete details see the fMRI methods section in \ref{TODO}.

\begin{figure}[tp]
	\label{fig:task}
	\fitfigure{f1task}
	\caption{Depiction of the behavioral task. The top is the (passive) classical conditioning participants learn the reward categories.  The bottom is the active abstract two-choice category learning.}
\end{figure}

\begin{figure}[tp]
	\label{fig:II}
	\fitfigure{f3II}
    % TODO -- replace with the good plot; find the good plot again.
	\caption{A diagram of sinusoidal grating distributions for an information integration (II) category.  As II categories span the diagonal of the gratings parameter space (line width ($W$) and angle ($\theta$)) successful learning requires consideration of both dimensions preventing participants from solving the categorization problem with simple rule based strategies (e.g. if the lines are wide is category ``a'')}
\end{figure}

\subsubsection{Repeated results}


\newpage
\bibliography {bibmin}
%%%%%%%%%%%%%
\end{document}
%%%%%%%%%%%%%
